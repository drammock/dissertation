\chapter{Discussion}

\section{Relevance}
The ability to separate and manipulate prosodic dimensions of speech opens up new experimental paradigms in syntax-phonology interface research, pragmatics, auditory neuroscience, audiology, and psycholinguistics.  Similar to the groundbreaking research in cues and cue trading (where, \eg a link was found between perceived stop voicing and preceding vowel length), this research opens the door to a more refined understanding of the complex relationships among the acoustic dimensions of loudness, pitch, duration, and timing, on one hand, and the phonological phenomena of stress, focus, and intonation on the other hand.

\section{Future directions}
\begin{itm}
	\item{decomposing effects of prosodic dimensions}
	\begin{itm}
		\item{{\bfseries Experiment 5:} What are the relative contributions of duration, pitch, and intensity to talker intelligibility?  (Repeat of experiment 1 where only 1 or 2 of the dimensions are replaced via resynthesis.)}
	\end{itm}
	\item{talker ID studies}
	\begin{itm}
		\item{{\bfseries Experiment 6:} Can you train listeners to make a reliable talker ID distinction on the basis of prosody alone?  Can listeners reliably categorize B/C and B/A as different talkers?}
		\item{{\bfseries Experiment 7:} Can you train listeners to make a reliable talker ID distinction when prosody has been neutralized?  Can listeners reliably categorize C/B and A/B as different talkers?}
%		\item{{\bfseries Experiment 8:} a reasonable supposition is that lexical activation allows a listener to more easily characterize a talker with respect to prior experience (\eg the talker “sounds Southern” or “glottalizes /t/ in coda position”).  This should be testable with talkerID task with two males from dialect A and one talker from each of a few other dialects.  Make sure that the second one doesn't show up until several of the other talkers have been heard, and show (hopefully) that the second male from dialect A is mistaken for the first one.  The idea is to show that talkerID initially relies on the most obvious / highest level differences between talkers, and if listener expectations are manipulated such that they treat “dialect” as the salient level of difference, even if the second talker from dialect A is “closer” to some of the other talkers on various non-dialectal dimensions (\eg pitch range, stop voicing during closure, etc).}
	\end{itm}
\end{itm}

