\chapter{Discussion\label{chap:Discussion}}
% TODO: check for interaction between base intelligibility of training talker, and improvement across quartiles.

The principal findings of the experiments described in this thesis can be summarized as follows.  First, given a broad separation between prosodic and segmental factors, there appear to be individual differences in the strategies employed by talkers to achieve \term{sufficient discriminability} \citep{Lindblom1990}.  That is, some talkers (such as Talker~\ac{a} in these experiments) maintain a higher information density at the segmental level of their speech, making them relatively intelligible even when the low-frequency modulations of their speech do little to enhance the successful transmission of information.  In contrast, other talkers (such as Talker~\ac{b}) compensate for lower information density at the segmental level by manipulating the low-frequency modulations of speech — \ie, by altering their prosody to make the remaining cues more likely to be successfully perceived.
% TODO: all this makes sense in terms of balancing articulatory effort with rate of transfer etc etc (aylett turk)

The finding that individuals vary in their intelligibility strategies also helps clarify previous literature, regarding the diversity of acoustic predictors shown to correlate with intelligibility in various studies.  In particular, it raises questions regarding the generalizability of findings for studies where only one or a few talkers are taken to represent the population at large.     
% TODO: there is clear evidence that intelligibility is not predictable along a single continuum

The emerging picture is that listeners may have several listening strategies depending on which strategy their current interlocutor is employing, along with the type of environment, background noise, etc.  This picture relates to why familiarity works (since it takes time to learn what listening strategy maximizes info transfer for a given talker).

Perhaps no less important are the methodological contributions, 

the fact that distortion was relatively low...

% TODO: Summary of results

% TODO: relate to robust/precise.  Discuss whether patterns will generalize to other environments / types of noise / languages

\section{shortcomings}
% TODO: only 3 talkers
% TODO: intensity method could be better
% TODO: training didn't work

\section{Methodological lessons}
% TODO: discuss segmentation issues, esp. with regard to intensity scaling (should have mapped mean or peak instead of contour?)
% TODO: discuss breakdown of prosodic dimensions into intensity, duration, and pitch.  Why didn't do it in first place?  wanted to avoid conflicting cues
% TODO: discuss alternative method: neutralize prosody by setting everyone equal to the mean of each dimension
% TODO: discuss training failure
% TODO: measurement or control of the degree of distortion...  how much does it vary from sentence to sentence?

\section{Relevance}
%The ability to separate and manipulate prosodic dimensions of speech opens up new experimental paradigms in syntax-phonology interface research, pragmatics, auditory neuroscience, audiology, and psycholinguistics.  Similar to the groundbreaking research in cues and cue trading (where, \eg, a link was found between perceived stop voicing and preceding vowel length), this research opens the door to a more refined understanding of the complex relationships among the acoustic dimensions of loudness, pitch, duration, and timing, on one hand, and the phonological phenomena of stress, focus, and intonation on the other hand.

\section{Future directions}
% TODO: relative contributions of duration, pitch, and intensity
% TODO: what underlies tune-in / tune-out?  voice quality?  talker ID study.
% 

% a reasonable supposition is that lexical activation allows a listener to more easily characterize a talker with respect to prior experience (\eg, the talker “sounds Southern” or “glottalizes /t/ in coda position”).  This should be testable with talkerID task with two males from dialect A and one talker from each of a few other dialects.  Make sure that the second one doesn't show up until several of the other talkers have been heard, and show (hopefully) that the second male from dialect A is mistaken for the first one.  The idea is to show that talkerID initially relies on the most obvious / highest level differences between talkers, and if listener expectations are manipulated such that they treat “dialect” as the salient level of difference, even if the second talker from dialect A is “closer” to some of the other talkers on various non-dialectal dimensions (\eg, pitch range, stop voicing during closure, etc).
