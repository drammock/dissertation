\chapter{Discussion\label{chap:Discussion}}
% TODO: Summary of results

% TODO: relate to robust/precise.  Discuss whether patterns will generalize to other environments / types of noise

\section{Methodological lessons}
% TODO: discuss segmentation issues, esp. with regard to intensity scaling (should have mapped mean or peak instead of contour?)
% TODO: discuss breakdown of prosodic dimensions into intensity, duration, and pitch.  Why didn't do it in first place?  wanted to avoid conflicting cues
% TODO: discuss alternative method: neutralize prosody by setting everyone equal to the mean of each dimension
% TODO: discuss training failure
% TODO: measurement or control of the degree of distortion...  how much does it vary from sentence to sentence?

\section{Relevance}
%The ability to separate and manipulate prosodic dimensions of speech opens up new experimental paradigms in syntax-phonology interface research, pragmatics, auditory neuroscience, audiology, and psycholinguistics.  Similar to the groundbreaking research in cues and cue trading (where, \eg, a link was found between perceived stop voicing and preceding vowel length), this research opens the door to a more refined understanding of the complex relationships among the acoustic dimensions of loudness, pitch, duration, and timing, on one hand, and the phonological phenomena of stress, focus, and intonation on the other hand.

\section{Future directions}
% TODO: relative contributions of duration, pitch, and intensity
% TODO: what underlies tune-in / tune-out?  voice quality?  talker ID study.
% 

% a reasonable supposition is that lexical activation allows a listener to more easily characterize a talker with respect to prior experience (\eg, the talker “sounds Southern” or “glottalizes /t/ in coda position”).  This should be testable with talkerID task with two males from dialect A and one talker from each of a few other dialects.  Make sure that the second one doesn't show up until several of the other talkers have been heard, and show (hopefully) that the second male from dialect A is mistaken for the first one.  The idea is to show that talkerID initially relies on the most obvious / highest level differences between talkers, and if listener expectations are manipulated such that they treat “dialect” as the salient level of difference, even if the second talker from dialect A is “closer” to some of the other talkers on various non-dialectal dimensions (\eg, pitch range, stop voicing during closure, etc).
