\chapter{Background}
This chapter presents a brief overview of relevant literature on speech intelligibility, prosody, and talker familiarity.  Because the experiments described in this thesis involve speech obscured by background noise, particular attention is given to the perception of speech-in-noise.  The discussion of talker familiarity touches on both short-term familiarity (i.e., training and exposure studies) and long-term familiarity.  Finally, the discussion of prosody focuses on pitch, loudness, and duration as they relate to speech perception.

\section{Auditory masking of speech}
It is well-known that auditory masking is dependent on both the spectrotemporal characteristics of the masking sound as well as its relationship to the target sound.  \citet{Miller1947} puts it succinctly: “...the masking of speech [depends] on three characteristics of the masking sound: (1) its intensity relative to the intensity of the speech, (2) its acoustic spectrum, and (3) its temporal continuity” \citep[106]{Miller1947}.  Even the early experiments of \citeauthor{WegelLane1924} (which involve only pure-tone targets and maskers) reveal the importance of the \emph{relationship} between target and masker sounds, in their observation that masker tones close in frequency to the target tone are the best maskers, except when the frequencies are so similar as to cause beating \citep{WegelLane1924}.  
%Quantitative measurements of auditory masking began in the 1920s with the seminal experiments of \citet{WegelLane1924}, who described the masking of one pure-tone signal by a second pure-tone signal of different frequency.  Studies of speech masking soon followed, with the work of \citet{Miller1947}, \citet{Cherry1953}, Tolhurst \citep{BlackEtAl1953}, Hirsh \citep{HirshEtAl1954}, Stevens \citep{HawkinsStevens1950}, and others.  %TODO: say what they did
Since those early experiments, numerous aspects of the target-masker relationship have been investigated and found to impact the masking of speech.  In this section, those factors are discussed in four broad categories: those related to masker signal type, those related to the talker, those related to the listener, and those related to the target signal content.

\subsection{Energetic and informational masking}
Early masking experiments \citep[e.g.,][]{HawkinsStevens1950,Tolhurst1957,PollackPickett1958} were often performed with noise maskers: random (gaussian) signals, usually without any amplitude modulation (“stationary maskers”) or frequency shaping (“white noise maskers”).  Masking due to any such random signal is typically termed “energetic masking”, reflecting the idea that the masking is a consequence of elevated signal detection thresholds in the auditory filters of the cochlea \citep{DurlachEtAl2003,xxx}.  However, most everyday situations involve listening to speech in the presence of competing speech streams — a situation poorly modeled by stationary white noise maskers, since real speech is both amplitude modulated and frequency shaped (indeed, the frequency shaping is itself dynamic).  

When speech is used as a masker, the amplitude modulations in the masker speech create temporal variations in \snr\ of the signal that can offer “glimpses” of the target stream.  The existence of such glimpses make speech a (potentially) worse masker than stationary noise, since listeners can use them to reconstruct neighboring spans of the target stream that are less clearly heard (based on lexical knowledge, phonotactics, contextual probabilities, etc., \citep{xxx}).  However, glimpses can also occur with amplitude-modulated random noise maskers, and if noise maskers are both amplitude-modulated and frequency-shaped so as to be maximally comparable to speech maskers, what we find is that target perception accuracy is lower when masked by speech than by noise \citep[e.g.,][]{CarhartEtAl1969,LewisEtAl1988,SimpsonCooke2005}.  The “additional” masking present in speech maskers is termed “informational masking” (less commonly: “perceptual masking”), and is usually attributed to the information contained in the masker signal causing competition for language processing resources at higher levels of the auditory and language processing streams in the brain \citep{DurlachEtAl2003,xxx}.

\subsection{Multitalker listening}
One can simultaneously reduce glimpses and decrease the recognizability of background speech by increasing the total number of background speech streams (these are often called “(multitalker) babble maskers”).  Early work by \citet{Miller1947} using babble maskers found that the degree of masking increased as the number of competing speech signals increased.  However, more recent research suggests that indefinitely increasing the number of talkers does not necessarily increase masking: total masking seems to plateau around 6–8 background talkers and gradually recede back to the lower level of masking seen in stationary, frequency-shaped random noise maskers as the number of background talkers is increased further \citep{BrungartEtAl2001,SimpsonCooke2005}.  This finding can be understood as the interaction between energetic and informational masking: %I interpret this finding as being consistent with the view that there are two competing aspects of a babble masker signal.  %: its patterns of spectrotemporal variation, and the information encoded by them.  More concretely, the amplitude modulation and spectral dynamics of natural speech allow occasional glimpsing of the target signal, effectively lowering masking in cases where the listener can recover whole words or phrases based on only partial suprathreshold access to the target signal \citep{Cooke2006}.  In contrast, the informational content of the background speech has the potential to be recognized by the listener and thereby introduce activations in the auditory and language processing systems of the brain that conflict with the target signal, and thus interfere with auditory word formation and/or lexical access.\citep{xxx}  
as the number of background talkers increases, the spectrotemporal glimpses tend to decrease (increasing energetic masking), while the informational content of any individual masker voice becomes increasingly obscured by competing background talkers (decreasing informational masking).  The “peak” masking seen with 6–8 talkers is thus a consequence of the interaction of these two aspects of the masker signals.

Other factors known to decrease informational masking include language mismatch between the target and masker speech \citep{RhebergenEtAl2005, VanEngenBradlow2007, WuEtAl2011, BrouwerEtAl2012}, perceived spatial separation of the target and masker voices \citep{BrungartSimpson2002, FreymanEtAl1999, FreymanEtAl2004, KiddEtAl2005b, JohnstoneLitovsky2006}, and dissimilarity between the target and masker voices \citep{Brungart2001}.  These and other factors will be discussed in following sections.


\subsection{Talker characteristics}
Setting aside real-world situations (accommodation,etc)...
  

\subsubsection{register, clear speech, lombard effects}
Lombard, unintentionally clear speech, intentionally clear speech, register.

\subsubsection{target-masker similarity}
Brungard 2001, male female same\citep{Brungart2001}

\subsubsection{measurable dimensions of speech correlated with intelligibility}
\begin{itm}
	\item{loudness}
	\item{speech rate}
	\item{reduction}
	\item{vowel space}
\end{itm}

\subsection{Listener characteristics}
\subsubsection{hearing loss}
In general human auditory system looks like xxx, with little variability.  However, hearing loss, cognitive decline

\begin{itm}
	\item{talker/listener dialect mismatch}
	\item{talker nonnative accent}
	\item{familiarity of target voice}
	\item{native vs non-native listeners \citep{CookeEtAl2008, CookeEtAl2010, BrouwerEtAl2012}}
	\item{It has been shown that foreign babble masks less well than native babble \citep{RhebergenEtAl2005,VanEngenBradlow2007,WuEtAl2011}.  At least in some cases, native-language babble masking is due to due to distracting lexical pop-out \citep{HoenEtAl2007}.  However, it is unclear whether differences are solely attributable to lexical interference from the native background speech, or whether other aspects of the native and foreign speech might be contributing to differential masking ability (e.g., familiar vs unfamiliar phones, rhythms, or intonation contours). Rhebergen et al's study is particularly problematic, since their native-language masker (Dutch) was time-scaled using PSOLA to match the speech rate of the (unmodified) foreign-language masker (Swedish).}
	\item{priming first few words of target utterance \citep{FreymanEtAl2004}}
	\item{hearing loss}
\end{itm}

\subsection{Signal content characteristics}
\begin{itm}
	\item{redundancy}
	\item{lexical effects \citep{HoenEtAl2007, BoulengerEtAl2010, BrouwerEtAl2012}}
	\item{lexical frequency: When the masker is a speech stream, high-frequency words are more likely than low-frequency words to pop out of the background and cause distraction (i.e., slow reaction time, or mishear target word for background competitor) \citep{BoulengerEtAl2010}.}
	\item{neighborhood density}
	\item{contextual probability / predictability.  For example, \citet{LewisEtAl1988} showed that words that are predictable from context are harder to mask, in that they are recoverable (or at least guessable) at lower SNRs than the same words in low-context sentences.  At the same time, words that are predictable from context are typically articulated less distinctively, as though the talker were balancing their own articulatory effort with expectations that the listener is paying attention and can recover some words more easily than others \citep{Wright2004}.  }
	\item{Lexical activation (\textsc{aka}, word recognition) also helps when learning to recognize talkers: if the talkers are speaking an unfamiliar language, it is harder to learn to distinguish their voices \citep{PerrachioneWong2007}.}
	\item{priming target words in a simultaneous unattended contralateral speech stream \citep{RivenezEtAl2006}}
	\item{priming target talker voice and spatial location (=familiarity) \citep{KiddEtAl2005a, KitterickEtAl2010}}
\end{itm}


%A synthesis of this research leads to the conclusion that virtually every aspect of a target or masker stimulus is potentially relevant to speech perception, including many facets of speech that are often difficult to balance or control for (e.g., talker voice quality, timing of glimpses in the masker envelope, lexical frequency and semantic predictability of target words, familiarity with the talker’s voice or dialect, etc).



\section{Prosody}

\subsection{Intensity}
\begin{itm}
	\item{stress}
	\item{glimpsing}
	\item{audibility}
	\item{word-by-word SNR}
\end{itm}

\subsection{Duration}
\begin{itm}
	\item{glimpsing}
	\item{stress}
	\item{non-native duration patterns reduce intelligibility\citep{QueneVanDelft2010}}
\end{itm}

\subsection{Pitch}
\begin{itm}
	\item{creaky voicing}
	\item{spacing between harmonics, masking in that freq region}
\end{itm}

\subsection{Regional variation in prosody}

\section{Familiarity}
\subsection{Training studies}
\subsection{Long-term familiarity studies}
Newman \& Evers: college prof study\citep{NewmanEvers2007}

\subsection{Priming studies}
