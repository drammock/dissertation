% % % % % % % % % % % % % % % % %
% % % % %  TITLE PAGE   % % % % %
% % % % % % % % % % % % % % % % %
\pagenumbering{roman}
\begin{spacing}{1.5}
\begin{titlepage}
	\begin{center}
	\thetitle \\ \vskip 2em
	\begin{tabular}[t]{c} \theauthor \end{tabular} \\	\vskip 3em
	A dissertation\\ submitted in partial fulfillment of the\\ requirements for the degree of\\ \vskip 2em
	Doctor of Philosophy\\ \vskip 3em
	University of Washington\\
	\thedate \\ \vskip 3em
	Reading Committee:\\ Richard A. Wright, Chair\\ Frederick J. Gallun\\ Sharon L. Hargus\\ Gina-Anne Levow\\ \vskip 3em
	Program Authorized to Offer Degree:\\ Linguistics
	\end{center}%
\end{titlepage}
\end{spacing}
\thispagestyle{empty}
\newpage

% % % % % % % % % % % % % % % %
% % % % %  ABSTRACT   % % % % %
% % % % % % % % % % % % % % % %
\begin{abstract}
\begin{spacing}{2}
Abstract text goes here.  350 words max.  Double space: abstract, dedication, acknowledgements, table of contents, and body of the manuscript, except for quotations as paragraphs, captions, items in tables, lists, graphs, charts. Single space: footnotes/endnotes, bibliographic entries, lists in appendices.
\end{spacing}
\end{abstract}
\newpage

% % % % % % % % % % % % % % % % % % %
% % % % %  TABLE OF CONTENTS  % % % %
% % % % % % % % % % % % % % % % % % %
\begin{spacing}{2}
\tableofcontents
\newpage

% % % % % % % % % % % % % % % % % % %
% % % % %  LIST OF FIGURES  % % % % %
% % % % % % % % % % % % % % % % % % %
\listoffigures
\newpage

% % % % % % % % % % % % % % % % % % %
% % % % %  LIST OF TABLES   % % % % %
% % % % % % % % % % % % % % % % % % %
\listoftables
\newpage

% % % % % % % % % % % % % % % % % % %
% % % % %  ACKNOWLEDGMENTS  % % % % %
% % % % % % % % % % % % % % % % % % %
\chapter*{Acknowledgments}
First and foremost I thank my advisor, mentor, and dissertation chair, Richard Wright.  For the last three years he has (among other things) challenged, inspired, critiqued, funded, encouraged, frustrated, and befriended me.  One of the main reasons I finished graduate school was that I truly enjoyed going to our meetings.  I also owe a large debt to the other members of my dissertation committee: Erick Gallun, Sharon Hargus, and Gina-Anne Levow, who guided me with keen questions along the way, and struggled through my inelegant draft prose to help me refine this work.  I would be remiss if I failed to also mention Pam Souza, who has been a great collaborator, and who showed me by her own example that science is at its best when we practice it in the service of humanity.  

My colleagues in the UW Linguistics Department have been fine companions, and I have profited from many stimulating discussions over the years.  Special thanks go to Bill McNeill, Julia Miller, Michael Goodman, Valerie Freeman, Lisa Tittle, Sarala Puthuval, Josh Crowgey, and most especially Darren Tanner and Steve Moran.  Most of my conversations with these folks had little to do with my dissertation research, but rather were a source of enrichment and distraction when it was sorely needed.  To the extent that I seem like a well-rounded scholar, it is often because of the little facets of knowledge that I gleaned from each of them.  Gus McGrath gets special mention here as well, for having so quickly transitioned from student to trainee to collaborator, and for doing so much of the dirty work that arises when creating spoken corpora.

I must also acknowledge my dear friends outside of academia, who for several years have tolerated my absence from parties, dinners, concerts, openings, performances and festivals, and yet still embraced me fondly when I did manage to show up once in a while.  My parents have been especially tolerant in this regard, and deserve thanks for so much more than I can describe here.

Along my winding path through education, I have had a number of gifted teachers who quietly inspired me to remain curious and to someday be as good a teacher as they were.  I have always wanted to acknowledge them publicly; this seems as good a place as any.  Thanks to (in chronological order): Sherry Brown (chemistry, grade 10), David Adams (calculus), Dennis Lamb (Greek and Roman literature), Bill Moody (cellular neurobiology), Larry BonJour (epistemology), Bill Talbott (ethics and social philosophy), David Knechteges (Chinese literature), Bi Nyan-Ping (Chinese language), Zev Handel (Chinese historical linguistics), and Chris Stecker (psychoacoustics).
\newpage

% % % % % % % % % % % % % % % % %
% % % % %  DEDICATION   % % % % %
% % % % % % % % % % % % % % % % %
\chapter*{Dedication}
To Sarala, \textit{sine qua non}.
\end{spacing}
