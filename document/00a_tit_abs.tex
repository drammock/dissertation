% % % % %  TITLE PAGE   % % % % %
\begin{titlepage}
	\begin{center}
	\thetitle \\ \vskip 2em
	\begin{tabular}[t]{c} \theauthor \end{tabular} \\	\vskip 3em
	A dissertation\\ submitted in partial fulfillment of the\\ requirements for the degree of\\ \vskip 2em
	Doctor of Philosophy\\ \vskip 3em
	University of Washington\\
	\thedate \\ \vskip 3em
	Reading Committee:\\ Richard A. Wright, Chair\\ Frederick J. Gallun\\ Sharon L. Hargus\\ Gina-Anne Levow\\ \vskip 3em
	Program Authorized to Offer Degree:\\ Linguistics
	\end{center}%
\end{titlepage}
\thispagestyle{empty}

% % % % %      ABSTRACT     % % % % %
\begin{abstract}
This thesis concerns the relationship between speech intelligibility and speech prosody, and the role that speech prosody plays in the perceptual advantage that comes from listening to a familiar talker.  A parallel corpus of 90 sentences (each spoken by three talkers of varying intelligibility) was used to create resynthesized stimuli in which fundamental frequency (\fo), intensity, and patterns of syllable duration were swapped between all possible pairs of talkers.  An additional 90 sentences were reserved for use as training stimuli.  Findings from speech\-/in\-/noise tasks suggest that the contribution of prosody to intelligibility varies considerably across talkers, evidenced by differences in sentence intelligibility after prosodic replacement from different prosodic donors.  In particular, high\-/intelligibility talkers need not have particularly “good” prosody if their intelligibility rests on articulatory strategies that emphasize robust segmental cues, and talkers with relatively good prosody may have low intelligibility due to non\-/prosodic factors.  \Ph{} acoustic analyses of the stimuli suggest that many acoustic measures index both prosodic and non\-/prosodic speech strategies, whereas acoustic measures that reflect the prosodic component of intelligibility are harder to find.  Utterance\-/final creaky voicing presents a particular challenge in this regard, due to its exaggeration of \fo-related measures.  The contribution of prosody to the familiar talker advantage remains unclear; listeners trained on different talkers showed different degrees of task adaptation during familiarization\slsh training, but were unable to generalize to a testing phase involving resynthesized talkers presented in random order.
%Abstract text goes here.  350 words max.  Double space: abstract, dedication, acknowledgements, table of contents, and body of the manuscript, except for quotations as paragraphs, captions, items in tables, lists, graphs, charts. Single space: footnotes/endnotes, bibliographic entries, lists in appendices.
\end{abstract}
