\documentclass[t]{beamer}
\usetheme{Boadilla} %\usetheme{Goettingen}
\usefonttheme{serif}
\usefonttheme{professionalfonts}
\usecolortheme{beaver}
\setbeamertemplate{itemize item}{•}
\setbeamertemplate{itemize subitem}{◦}
\setbeamertemplate{itemize subsubitem}{—}
\setbeamertemplate{section in toc}[default] % include section on outline slide
\setbeamertemplate{subsection in toc}[default] % include subsection on outline slide
\beamertemplatenavigationsymbolsempty % no bullets on outline slide
\setbeamercovered{transparent=30} % global setting for slide builds (enabled by the <1-2> codes after \item)

\usepackage{fontspec, setspace, tabu, booktabs, multirow, url, amsmath, colortbl}
\usepackage[sort&compress,super]{natbib}

% Use these with \footnotemark, \footnotetext & \bibentry if you want references to show up on each slide
%\usepackage{bibentry} 
%\nobibliography*

% misc hacks
\def\newblock{\hskip .11em plus .33em minus .07em}
\definecolor{lgray}{gray}{0.7}
\newenvironment{hl}{\usebeamercolor[fg]{frametitle}}

% fonts
\setmainfont[Numbers={Lining}]{Linux Libertine O} \setsansfont{Linux Biolinum O} \setmonofont{Linux Libertine Mono O}
%\newfontfamily{\charis}{Charis SIL}
%\newenvironment{ipa}{\large\charis}{}%
\renewcommand\url{\begingroup \def\UrlLeft{}\def\UrlRight{}\urlstyle{same}\Url} % set URLs in same font as surrounding text

% replacement for itemize environment
\newenvironment{itm}{%
	\setlength{\leftmargini}{0.5em}%
	\setlength{\leftmarginii}{1em}%
	\begin{itemize}}{\end{itemize}%
}

% misc abbreviations
\newcommand{\term}[1]{“#1”} % first occurrence of technical terms
\newcommand{\ac}[1]{\textsc{#1}} % acronyms % \newcommand{\ac}[1]{\MakeUppercase{#1}}
\newcommand{\lat}[1]{\textit{#1}} % foreign words and abbreviations
\newcommand{\psola}{\ac{psola}™}
\newcommand{\fo}{ƒ₀} % font problems? might work as {\(f_0\)}
\newcommand{\eg}{\lat{e.g.}}
\newcommand{\ie}{\lat{i.e.}}
\newcommand{\etseq}{\lat{et seq}}
\newcommand{\etc}{\lat{etc}}
\newcommand{\intal}{\lat{inter alia}}
\newcommand{\ph}{\lat{post hoc}}
\newcommand{\Ph}{\lat{Post hoc}}
\newcommand{\aka}{\textsc{aka}}
\newcommand{\vs}{\lat{vs.}}
\newcommand{\vv}{\lat{vice versa}}
\newcommand{\perse}{\lat{per se}}
\newcommand{\slsh}{/‌} % contains a zero-width non-joiner after the slash, to allow line breaking

% Automatically insert overview/roadmap slide at start of each section
\AtBeginSection[]{%
\begin{frame}<beamer>
    \frametitle{Overview}
    \tableofcontents[currentsection]
  \end{frame}
}

\title[Prosody, intelligibility \& familiarity]{Prosody, intelligibility and familiarity in speech perception}
%\subtitle[short subtitle]{long subtitle}
\author[McCloy]{Daniel~McCloy} 
\institute[UW]{Department of Linguistics, University of Washington}
\date[2013.05.31]{2013 May 31}
\subject{Linguistics, Phonetics, Speech Perception}

\begin{document}

%===== TITLEPAGE =====%
\frame{\titlepage}


%===== TABLE OF CONTENTS =====%
\begin{frame}{Overview}
	\tableofcontents{}
\end{frame}

%===== MOTIVATION =====%
\section{Motivation}

\subsection{Why study intelligibility?}
\begin{frame}{Why study intelligibility?}
	\begin{itm}
		\item<1-3> \emph{What is the same across talkers?}
		\begin{itm}
			\item Auditory system (\ie, target–masker language similarity)
		\end{itm}
		\item[]
		\item<2-3> \emph{What is different across talkers?}
		\begin{itm}
			\item Applications to assistive hearing device design (acoustics, signal processing)
			\item Applications to speech therapy, pedagogy (articulation)
		\end{itm}
		\item[]
		\item<3> \emph{What is different across groups?}
			\begin{itm}
				\item Cross-linguistic differences in perception
				\begin{itm}
					\item Insight into “active” phonological features
				\end{itm}
			\end{itm}
	\end{itm}
\end{frame}

\subsection{Why study prosody?}
\begin{frame}{Why study prosody?}
	\begin{itm}
		\item<1-3> Hearing aids: prosodic changes easier than segmental changes
		\item[]
		\item<2-3> Natural distinction: segmental \vs\ supersegmental
		\begin{itm}
			\item Duration, \fo, and intensity not phonemic in English as spoken here\\(secondary cues)
		\end{itm}
		\item[]
		\item<3> Bottom-up approach using acoustic predictors: conflicting results
	\end{itm}
\end{frame}

%===== BACKGROUND =====%
\section{Background}

\begin{frame}{What impacts intelligibility?}
	\begin{columns}
		\begin{column}{0.5\textwidth}
		\begin{itm}
			\item<1-2> Language user properties
			\begin{itm}
				\item Speech style
				\item Listener experience with language/dialect
				\item Intrinsic individual differences?
				\item Gender?
			\end{itm}
		\end{itm}
		\end{column}
		\begin{column}{0.5\textwidth}
		\begin{itm}
			\item<2> Signal properties
			\begin{itm}
				\item Speech rate
				\item Vowel space size
				\item Pitch range
			\end{itm}
		\end{itm}
		\end{column}
	\end{columns}
	
\end{frame}

%===== METHODS =====%
\section{Methodology}

\begin{frame}{General approach}
	\begin{itm}
		\item<1-3> Top-down \vs\ bottom-up
		\begin{itm}
			\item<2-3> Not a perfect split:
			\begin{itm}
				\item Vowel reduction (formant values) affected by phrasal accent
				\item Consonant reduction (spirantization, glottalization, flapping) affected by prosody
				\item Duration is secondary cue to phonemic distinctions
				\item Pitch is (subtly) affected by vocal tract constrictions (\ie, consonants)
			\end{itm}
		\end{itm}
		\item[]
		\item<3> \emph{Why use resynthesis?}
		\begin{itm}
			\item Advantage: Less work than parametric synthesis; natural-sounding voices
			\item Disadvantage: risk of distortion; real voices!
		\end{itm}
	\end{itm}
\end{frame}

\begin{frame}{Stimulus design}
	\begin{itm}
		\item Careful recordings of parallel sentences (head mounted mics, low noise floor, coached to standardize prosody)
		\item Five \ac{pnw} male talkers included in corpus; three selected for resynthesis
		\item Duration, intensity \& pitch swapped between all possible pairs of talkers
	\end{itm}
\end{frame}

\begin{frame}{Resynthesis}
	\begin{itm}
		\item \psola{} algorithm as implemented in Praat
		\item 
	\end{itm}
\end{frame}

%===== RESULTS =====%
\section{Results}

%===== DISCUSSION =====%
\section{Discussion}

%===== REFERENCES =====%
% Don't give this its own section, unless you want an auto-generated return to the outline slide just before it
%\begin{frame}{References}
%	\tiny
%	\bibliography{bibFileName}
%	\bibliographystyle{apa-good}		
%\end{frame}


\subsection{Acknowledgments} % include if you want on the outline / in PDF bookmarks
\begin{frame}{Acknowledgments}
	\begin{itm}
		\centering
		\item[] Thanks to:
		\item[] 
		\item[] Richard~Wright, Sharon~Hargus, Erick~Gallun, Gina-Anne~Levow, Pam~Souza
		\item[] 
		\item[] Jennifer~Haywood, Gus~McGrath, Steve~Moran, Darren~Tanner, members~of~the~UW~Phonetics~Lab
		\item[] 
		\item[] Sherry~Brown, David~Adams, Dennis~Lamb, Bill~Moody, Larry~BonJour, Bill~Talbott, David~Knechteges, Bi~Nyan-Ping, Zev~Handel, Chris~Stecker, KC~Lee
		\item[] 
		\item[] Sarala~Puthuval
	\end{itm}
\end{frame}

\end{document}

%\begin{frame}{TableExample}
%	\begin{itm}
%		\item<1-2> foo
%		\item[] % Typically you'll want a blank line before and after the table item
%		\item[]<2> % [] suppresses bullet for table item
%		{\small
%		\begin{tabu} to 0.9\textwidth {l l X}
%			\toprule
%			\arrayrulecolor{lgray}
%			foo & foo & foo\\ \midrule
%			foo & foo & foo\\ \midrule
%			foo & foo & foo\\
%			\arrayrulecolor{black}\bottomrule
%		\end{tabu}
%		}%small
%		\item[] % Typically you'll want a blank line before and after the table item
%		\item<3> foo takeaway
%	\end{itm}
%\end{frame}

