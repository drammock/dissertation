\documentclass[t]{beamer}
\usetheme{Boadilla} %\usetheme{Goettingen}
\usefonttheme{serif}
\usefonttheme{professionalfonts}
\usecolortheme{beaver}
\setbeamertemplate{itemize item}{•}
\setbeamertemplate{itemize subitem}{◦}
\setbeamertemplate{itemize subsubitem}{—}
\setbeamertemplate{section in toc}[default] % include section on outline slide
\setbeamertemplate{subsection in toc}[default] % include subsection on outline slide
\beamertemplatenavigationsymbolsempty % no bullets on outline slide
\setbeamercovered{transparent=30} % global setting for slide builds (enabled by the <1-2> codes after \item)

\usepackage{fontspec, setspace, tabu, booktabs, multirow, url, amsmath, colortbl}
\usepackage[sort&compress,super]{natbib}

% Use these with \footnotemark, \footnotetext & \bibentry if you want references to show up on each slide
%\usepackage{bibentry} 
%\nobibliography*

% misc hacks
\def\newblock{\hskip .11em plus .33em minus .07em}
\definecolor{lgray}{gray}{0.7}
\newenvironment{hl}{\usebeamercolor[fg]{frametitle}}

% fonts
\setmainfont[Numbers={Lining}]{Linux Libertine O} \setsansfont{Linux Biolinum O} \setmonofont{Linux Libertine Mono O}
%\newfontfamily{\charis}{Charis SIL}
%\newenvironment{ipa}{\large\charis}{}%
\renewcommand\url{\begingroup \def\UrlLeft{}\def\UrlRight{}\urlstyle{same}\Url} % set URLs in same font as surrounding text

% replacement for itemize environment
\newenvironment{itm}{%
	\setlength{\leftmargini}{0.5em}%
	\setlength{\leftmarginii}{1em}%
	\begin{itemize}}{\end{itemize}%
}

% Automatically insert overview/roadmap slide at start of each section
\AtBeginSection[]{%
\begin{frame}<beamer>
    \frametitle{Overview}
    \tableofcontents[currentsection]
  \end{frame}
}

\title[myDissDefense]{Prosody, intelligibility and familiarity in speech perception}
%\subtitle[short subtitle]{long subtitle}
\author[McCloy]{Daniel~McCloy\inst{1}} 
\institute[UW]{%
\inst{1 } Department of Linguistics, University of Washington
}
\date[2013.05.31]{2013 May 31}
\subject{Linguistics, Phonetics, Speech Perception}

\begin{document}

%===== TITLEPAGE =====%
\frame{\titlepage}


%===== TABLE OF CONTENTS =====%
\begin{frame}{Overview}
	\tableofcontents{}
\end{frame}

%===== MOTIVATION =====%
\section{Motivation}

\begin{frame}{Why study intelligibility?}
	\begin{itm}
		\item<1-3> \emph{What is the same across talkers?}
		\begin{itm}
			\item Auditory system (\ie, target–masker language similarity)
		\end{itm}
		\item<2-3> \emph{What is different across talkers?}
		\begin{itm}
			\item Applications to assistive hearing device design (acoustics, signal processing)
			\item Applications to speech therapy, pedagogy (articulation)
		\end{itm}
		\item<3> \emph{What is different across groups?}
			\item Cross-linguistic differences in perception
			\begin{itm}
				\item Insight into “active” phonological features
			\end{itm}
	\end{itm}
\end{frame}

\begin{frame}{Why study prosody?}
	\begin{itm}
		\item<1> Hearing aids: prosodic changes easier than segmental changes
		\item<2> Natural distinction: segmental \vs\ supersegmental
		\begin{itm}
			\item Duration, \fo, and intensity not phonemic in English spoken here (secondary cues)
		\end{itm}
		\item<3> Bottom-up approach using acoustic predictors: conflicting results
	\end{itm}
\end{frame}

%===== BACKGROUND =====%
\section{Background}

\begin{frame}{What impacts intelligibility?}
	\begin{columns}
		\begin{column}{0.5\textwidth}
		\begin{itm}
			\item User properties
			\begin{itm}
				\item Speech style
				\item Listener experience with language/dialect
				\item Intrinsic individual differences?
				\item Gender?
			\end{itm}
		\end{itm}
		\end{column}
		\begin{column}{0.5\textwidth}
		\begin{itm}
			\item Signal properties
			\begin{itm}
				\item Speech rate
				\item Vowel space size
				\item Pitch range
			\end{itm}
		\end{itm}
		\end{column}
	\end{columns}
	
\end{frame}

%===== METHODS =====%
\section{Methodology}

%===== RESULTS =====%
\section{Results}

%===== DISCUSSION =====%
\section{Discussion}



%===== BACKGROUND =====%
\section{Background}

	\begin{frame}{BackgroundIntroSlide}
		\begin{itm}
			\item foo
			\begin{itm}
				\item foo
				\item foo
			\end{itm}
			\item[] % blank line
			\item<2> foo
			\begin{itm}
				\item foo
				\item foo
			\end{itm}
		\end{itm}
	\end{frame}

\subsection{BackgroundSubsection}

	\begin{frame}{BackgroundSubsectionSlide}
		\begin{itm}
			\item<1-2> foo
			\item[] % Typically you'll want a blank line before and after the table item
			\item[]<2> % [] suppresses bullet for table item
			{\small
			\begin{tabu} to 0.9\textwidth {l l X}
				\toprule
				\arrayrulecolor{lgray}
				foo & foo & foo\\ \midrule
				foo & foo & foo\\ \midrule
				foo & foo & foo\\
				\arrayrulecolor{black}\bottomrule
			\end{tabu}
			}%small
			\item[] % Typically you'll want a blank line before and after the table item
			\item<3> foo takeaway
		\end{itm}
	\end{frame}

%===== SECTION TWO =====%
\section{SectionTwo}

\subsection{SubsectionTwo}
	\begin{frame}{SectionTwoSlide}
		\begin{itm}
			\item foo
			\item foo
			\begin{itm}
				\item[] % blank line
				\item foo 
				\item[] % blank line
				\item foo
			\end{itm}
		\end{itm}	
	\end{frame}

%===== REFERENCES =====%
% Don't give this its own section, unless you want an auto-generated return to the outline slide just before it
%\begin{frame}{References}
%	\tiny
%	\bibliography{bibFileName}
%	\bibliographystyle{apa-good}		
%\end{frame}


\subsection{Acknowledgments} % include if you want on the outline / in PDF bookmarks
\begin{frame}{Acknowledgments}
	\begin{itm}
		\item Many thanks to the organizers and participants of XXX!
		\item[] % blank line
		\item Thanks to my collaborators: XXX
		\item[] % blank line
		\item This project was (partially) supported by funding from XXX 
	\end{itm}
\end{frame}

\end{document}
