\chapter{Research questions\label{chap:Questions}}

\section{Overview}
Most of the research on intelligibility described in Chapter~\ref{chap:Background} is of one of two types, which I will refer to as the \term{modeling approach} and the \term{manipulation approach}.  Studies using the modeling approach examine two or more speaker populations and attempt to correlate differences in intelligibility with differences in other attributes of the speakers.  Studies of this type include \citet{BradlowEtAl1996, HazanMarkham2004, Neel2008, McCloyEtAl2013}.  A major shortcoming of this approach is that naturalistic speech varies among many dimensions simultaneously, so it is difficult to segregate the contributions of the various dimensions being measured.  It is also difficult to know whether variation on a given dimension (\eg, gender) is in fact genuinely contributing to differences in intelligibility, or merely indexing some other unmeasured dimension (\eg, pitch range) that is in fact what listeners are making use of in the perceptual task. 

In contrast, studies using the manipulation approach take a dimension believed to be relevant to the speech perception task, and create stimuli with a known manipulation along that dimension.  Examples of the manipulation approach include \citet{xxx}.  Such studies allow researchers to segregate the contribution of different dimensions of speech (\citet{KiddEtAl2005a} is a good example of this), but can be criticized for being too far removed from real-world scenarios of speech perception, either because the speech content is too controlled (as in \citeauthor{KiddEtAl2005a}’s use the \ac{crm} sentences), or the experimental conditions are too contrived (\eg, knowing that the target speech will occur in the fourth of the thirteen pairs of talkers, which start at 800 ms intervals).

The experiments here take a different approach, top down instead of bottom up.

\section{Research questions}


\subsection{How does prosody relate to intelligibility?}
\begin{itm}
	\item{Can we increase a difficult talker’s intelligibility simply by manipulating duration, intensity, and pitch?}
	\item{What are the relative contributions of the three aspects of prosody?}
\end{itm}

\subsection{What is the relationship between prosody and talker familiarity?}
\begin{itm}
	\item{Will a novel talker be more intelligible if his prosody has been replaced (via resynthesis) with the prosody of a familiar talker?  (compared to a novel talker with his own prosody, or with the prosody of a different novel talker)}
	\item{Will the familiarity advantage persist even when the familiar talker’s prosody is replaced?}
\end{itm}


\section{Experimental designs}
\subsection{Experiments testing the Effect of Prosody on Intelligibility}
{\bfseries Experiment 1:} Can we shift the intelligibility of a talker by replacing his prosody?\\
{\bfseries Prediction:} yes\\
{\bfseries Test:} C/A > C/B > C/C

{\bfseries Experiment 2:} Does the non-prosodic portion of the signal contribute to intelligibility differences?\\
{\bfseries Prediction:} yes\\
{\bfseries Test:} A/C > B/C 

\subsection{Training experiments}
{\bfseries Interim Experiment:} How much training is needed to see a talker familiarity advantage with stimuli of this type?  Does the advantage level off after a certain number of sentences? \textit{Note: may be able to skip this if sufficiently detailed findings in the literature turn up; so far I haven’t found them.}\\compare/titrate:\\C/C trained on C/C\\C/C trained on A/A\\A/A trained on A/A, etc.

{\bfseries Experiment 3:} Do stimuli from novel talkers (but resynthesized to have familiar-talker prosody) get a familiarity advantage?\\
{\bfseries Prediction:} yes\\
{\bfseries Test:} C/A (trained on A/A) > C/A (trained on B/B)

{\bfseries Experiment 4:} Is there value in the non-prosodic portion of the signal with regard to familiarity training?\\
{\bfseries Prediction:} yes\\
{\bfseries Test:} C/A (trained on C/C) > C/A (trained on B/B)
