\chapter{Research questions\label{chap:Questions}}

\section{Overview}
Most of the research on intelligibility described in Chapter~\ref{chap:Background} is of one of two types, which I will refer to as the \term{modeling approach} and the \term{controlled manipulation approach}.  Studies using the modeling approach examine two or more speaker populations and attempt to correlate differences in intelligibility with differences in other attributes of the speakers.  Studies of this type include \citet{BondMoore1994, BradlowEtAl1996, HazanMarkham2004, Neel2008}.  A major shortcoming of this approach is that naturalistic speech varies among many dimensions simultaneously, so it is difficult to segregate the contributions of the various dimensions being measured.  It is also difficult to know whether variation on a given dimension (\eg, gender) is in fact genuinely contributing to differences in intelligibility, or merely indexing some other unmeasured dimension (\eg, pitch range) that is in fact what listeners are making use of in the perceptual task.

In contrast, studies using the controlled manipulation approach take a dimension believed to be relevant to the speech perception task, and create stimuli with a known manipulation along that dimension.  Examples of the manipulation approach include \citet{KiddEtAl2005a, KitterickEtAl2010, DubnoEtAl2012}.  Such studies allow researchers to segregate the contribution of different dimensions of speech (\citet{KitterickEtAl2010} is a good example of this), but can be criticized for being too far removed from real-world scenarios of speech perception, either because the speech content is too controlled (\ie, the \ac{crm} corpus), or the experimental conditions are too contrived (\eg, knowing that the target speech will occur in the fourth of the thirteen pairs of talkers, which start at 800 ms intervals, as in \citealt{KitterickEtAl2010}).

The experiments proposed below are most similar to the modeling approach, but with more of a \term{top-down} approach to predictor selection.  In other words, rather than starting with low-level measurables (\eg, mean pitch, vowel space expansion, \etc), these experiments begin with a broad distinction between segmental and suprasegmental aspects of speech, and attempt to determine the contribution of each to the speech intelligibility.  This approach is somewhat analagous to the temporal distinction between \term{envelope} and \term{fine structure} commonly seen in the audiology literature \citep[\eg][]{Rosen1992}, although the emphasis here is less on temporal scale and more on the linguistic contrast between prosody (patterns of duration, intensity, pitch, and to some extent phonation) and segmental content (everything else).  The experiments described here also bear some similarity to cue weighting studies: insofar as they concern the relative importance of prosodic \vs\ segmental aspects of the signal, this constitutes an implicit comparision of prosodic cues (taken as a group) \vs\ segmental cues (taken as a group).

\section{Research questions}
The first research question addressed in this thesis is: {\em how does prosody relate to intelligibility?}  In other words, are differences in intelligibility between talkers attributable to their intonation and rhythm (at least in part)?  A related question (with relevance in audiological contexts) is whether an unintelligible talker can be made more intelligible through prosodic changes alone (without changes to segmental phonetic features like formant transitions, lenited consonants, missing release bursts, \etc).

% What are the relative contributions of the three aspects of prosody (intensity, pitch, duration)?

The second research question concerns the intersection of prosody, intelligibility, and talker familiarity: {\em to what extent is the familiarity advantage relying on prosodic aspects of the talker’s speech?}  In other words, when a  listener is sufficiently attuned to a talker’s voice such that they receive a perceptual advantage from that familiarity, what is it about the talker’s voice that they are \term{tuning in} to?  More concretely, is a novel talker more intelligible if his prosody mimics the prosody of a familiar talker?  Or conversely, will the familiarity advantage persist even when the familiar talker’s prosody changes to mimic the prosody of a novel talker?

\section{Experimental designs}
Both of the research questions above can be investigated using \psola{} resynthesis \citep{MoulinesCharpentier1990}, which allows manipulation of the pitch, duration, and intensity of speech.  
% in a nutshell: summarize the overall methodology (see Chapter~\ref{chap:Methods} for details).
% describe the contrasts we are interested in (A/C vs A/B vs A/A), foreshadowing the results.
% overview of the four experiments:
% 1. Can we shift the intelligibility of a talker by replacing his prosody? 
% 2. What is the contribution of non-prosodic aspects of speech, with regard to intelligibility?
% 3. Does the talker familiarity advantage persist with prosodic replacement?   

\exclude{
\subsection{Experiments testing the Effect of Prosody on Intelligibility}
{\bfseries Experiment 1:} Can we shift the intelligibility of a talker by replacing his prosody?\\
{\bfseries Prediction:} yes\\
{\bfseries Test:} C/A > C/B > C/C

{\bfseries Experiment 2:} Does the non-prosodic portion of the signal contribute to intelligibility differences?\\
{\bfseries Prediction:} yes\\
{\bfseries Test:} A/C > B/C 

\subsection{Training experiments}
{\bfseries Interim Experiment:} How much training is needed to see a talker familiarity advantage with stimuli of this type?  Does the advantage level off after a certain number of sentences? \textit{Note: may be able to skip this if sufficiently detailed findings in the literature turn up; so far I haven’t found them.}\\compare/titrate:\\C/C trained on C/C\\C/C trained on A/A\\A/A trained on A/A, etc.

{\bfseries Experiment 3:} Do stimuli from novel talkers (but resynthesized to have familiar-talker prosody) get a familiarity advantage?\\
{\bfseries Prediction:} yes\\
{\bfseries Test:} C/A (trained on A/A) > C/A (trained on B/B)

{\bfseries Experiment 4:} Is there value in the non-prosodic portion of the signal with regard to familiarity training?\\
{\bfseries Prediction:} yes\\
{\bfseries Test:} C/A (trained on C/C) > C/A (trained on B/B)
}
