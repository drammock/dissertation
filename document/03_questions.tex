\chapter{Research questions}

\section{Research questions}
\subsection{How does prosody relate to intelligibility?}
\begin{itm}
	\item{Can we increase a difficult talker’s intelligibility simply by manipulating duration, intensity, and pitch?}
	\item{What are the relative contributions of the three aspects of prosody?}
\end{itm}

\subsection{What is the relationship between prosody and talker familiarity?}
\begin{itm}
	\item{Will a novel talker be more intelligible if his prosody has been replaced (via resynthesis) with the prosody of a familiar talker?  (compared to a novel talker with his own prosody, or with the prosody of a different novel talker)}
	\item{Will the familiarity advantage persist even when the familiar talker’s prosody is replaced?}
\end{itm}


\section{Experimental designs}
\subsection{Experiments testing the Effect of Prosody on Intelligibility}
{\bfseries Experiment 1:} Can we shift the intelligibility of a talker by replacing his prosody?\\
{\bfseries Prediction:} yes\\
{\bfseries Test:} C/A > C/B > C/C

{\bfseries Experiment 2:} Does the non-prosodic portion of the signal contribute to intelligibility differences?\\
{\bfseries Prediction:} yes\\
{\bfseries Test:} A/C > B/C 

\subsection{Training experiments}
{\bfseries Interim Experiment:} How much training is needed to see a talker familiarity advantage with stimuli of this type?  Does the advantage level off after a certain number of sentences? \textit{Note: may be able to skip this if sufficiently detailed findings in the literature turn up; so far I haven’t found them.}\\compare/titrate:\\C/C trained on C/C\\C/C trained on A/A\\A/A trained on A/A, etc.

{\bfseries Experiment 3:} Do stimuli from novel talkers (but resynthesized to have familiar-talker prosody) get a familiarity advantage?\\
{\bfseries Prediction:} yes\\
{\bfseries Test:} C/A (trained on A/A) > C/A (trained on B/B)

{\bfseries Experiment 4:} Is there value in the non-prosodic portion of the signal with regard to familiarity training?\\
{\bfseries Prediction:} yes\\
{\bfseries Test:} C/A (trained on C/C) > C/A (trained on B/B)
