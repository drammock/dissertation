% % % % % % % % % % % % % % % % %
% % % % %  INTRODUCTION   % % % %
% % % % % % % % % % % % % % % % %
\pagenumbering{arabic}
\chapter{Introduction}
\section[Overview]{Overview of the thesis \label{sec:Overview}}
Some introductory text here...

\section[Abbreviations \& acronyms]{Abbreviations and acronyms used in the thesis \label{sec:Abbr}}
Following is a table of acronyms and abbreviations used in this thesis.  I have endeavored to spell out each acronym at the point it is first used in the document, but I cannot be certain that I have succeeded in this task, and in any event there will always be readers who choose to skip certain sections, or whose memory for new terminology is imperfect.  I hope this list is useful to them.

\begin{table}
	\caption[Abbreviations and acronyms]{Abbreviations and acronyms used in the thesis \label{tab:Abbr}}
	\centering
	\begin{tabu} to \textwidth [c]{X[c] X[5]}
		\toprule
		\rowfont[c]{\bfseries} Abbreviation & Explanation\\
		\midrule
		\ac{crm} & Coordinate response measure (a corpus of parallel sentences of the form “Ready ⟨\textsc{callsign}⟩ go to ⟨\textsc{color}⟩ ⟨\textsc{digit}⟩ now”).  In a typical competing speech usage, the listener is told to attend to the speech stream containing a particular callsign, and repeat the color and digit from that stream.  See \citet{BoliaEtAl2000} for details.\\
		\fo & Fundamental frequency (of speech)\\
		F2×F1 & The acoustic space defined by the center frequencies of the first and second formants of sonorant speech sounds (usually vowels)\\
		L1, L2, etc. & First (native) language, second language, etc.\\
		\ac{rms} & Root mean square (a method of measuring sound power)\\
		\ac{snr} & Signal-to-noise ratio ()\\
		\ac{srt} & Speech reception threshold ()\\
		\bottomrule
	\end{tabu}
\end{table}
