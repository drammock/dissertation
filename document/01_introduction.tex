% % % % % % % % % % % % % % % % %
% % % % %  INTRODUCTION   % % % %
% % % % % % % % % % % % % % % % %
\chapter{Introduction}
\section[Overview]{Overview of the thesis \label{sec:Overview}}
This thesis is concerned with two main questions.  The first question — {\emph what is the relationship between prosody and intelligibility?} — is investigated using careful resynthesis of read sentences, which are then presented in a listening task.  The second question — {\emph what is the relationship between prosody and the familiar talker advantage?} — is investigated in a training task utilizing the same stimuli created to address the first question.  The first part of the thesis comprises a review of relevant literature (Chapter~\ref{chap:Background}), followed by a formal statement of research questions (Chapter~\ref{chap:Questions}).  A detailed description of methods (including stimulus creation, experimental design, and data analysis) follows in Chapter~\ref{chap:Methods}.  Results of the two experiments are presented in Chapter~\ref{chap:Results}, followed by a general discussion in Chapter~\ref{chap:Discussion}.

\section[Abbreviations \& acronyms]{Abbreviations and acronyms used in the thesis \label{sec:Abbr}}
Following is a table of acronyms and abbreviations used in this thesis.  I have endeavored to spell out each abbreviation or acronym at the point it is first used in the document, but all of them are included here as well for convenient reference.

\begin{table}
	\caption[Abbreviations and acronyms]{Abbreviations and acronyms used in the thesis \label{tab:Abbr}}
	\centering
	\begin{tabu} to \textwidth [c]{X[c m] X[5 m]}
		\toprule
		\everyrow{\midrule}
		\rowfont[c]{\bfseries} Abbreviation & Explanation\\
		\taburulecolor{ltgray}
		{dB} & Decibels (a dimensionless logarithmic unit commonly used to express the intensity or power of a sound, relative to a known reference value).\\ 
		{dB \ac{hl}} & Intensity in decibels relative to frequency-specific averages representative of normal-hearing humans \citepalias{ansi2004}.  “\ac{hl}” stands for “hearing level”.\\
		{dB \ac{spl}} & Intensity in decibels relative to 20 micropascals (μPa) \citepalias{ansi1994}.  “\ac{spl}” stands for “sound pressure level”.\\
		\ac{crm} & Coordinate response measure (a corpus of parallel sentences of the form “Ready ⟨\textsc{callsign}⟩ go to ⟨\textsc{color}⟩ ⟨\textsc{digit}⟩ now”).  In a typical competing speech usage, the listener is told to attend to the speech stream containing a particular callsign, and to repeat the color and digit from that stream.  See \citet{BoliaEtAl2000} for details.\\
		\fo & Fundamental frequency (of speech).  Often represented as F₀ or F0, the form \fo{} is preferred because it emphasizes that the fundamental frequency is not a formant (which are typically represented as F1, F2, \etc).  The form f0 is an acceptable substitute when glyph choice is restricted to \ac{ascii} characters.\\
		F2×F1 & The acoustic space defined by the center frequencies of the first and second formants of sonorant speech sounds (usually vowels).\\
		\ac{hl} & See {dB \ac{hl}}.\\
		L1, L2, \etc. & First (native) language, second language, \etc.\\
		\ac{ltas} & Long-term average spectrum.\\
		\ac{mcmc} & Markov-chain Monte Carlo (an algorithm for sampling from probability distributions; useful for simulating the parameters of mixed models for, \eg, obtaining estimated \textit{p}-values).\\
		\psola & Pitch synchronous overlap and add (a method for time-domain resynthesis of speech; see \citealt{CharpentierMoulines1988, MoulinesCharpentier1990}).  \psola{} is a trademark of France Télécom.\\
		\ac{rms} & Root mean square, a measure of sound magnitude, defined as: \(\sqrt{\frac{1}{n}(x_1^2+x_2^2+\cdots+x_n^2)}\), where \(x_1 \ldots x_n\) are the samples of a digital signal.\\
		\ac{snr} & Signal-to-noise ratio.\\
		\ac{spl} & See {dB \ac{spl}}.\\
		\everyrow{}
		\ac{srt} & Speech reception threshold (the intensity level or \ac{snr} at which a listener achieves 50\% correct in a speech recognition task).\\
		\taburulecolor{black}
		\bottomrule
	\end{tabu}
\end{table}
