% % % % % % % % % % % % % % % % %
% % % % %  INTRODUCTION   % % % %
% % % % % % % % % % % % % % % % %
\chapter{Introduction}
\section[Overview]{Overview of the thesis \label{sec:Overview}}
Some introductory text to be written here after the rest of the dissertation is done.

\section[Abbreviations \& acronyms]{Abbreviations and acronyms used in the thesis \label{sec:Abbr}}
Following is a table of acronyms and abbreviations used in this thesis.  I have endeavored to spell out each abbreviation or acronym at the point it is first used in the document, but all of them are included here as well for convenient reference.

\begin{table}
	\caption[Abbreviations and acronyms]{Abbreviations and acronyms used in the thesis \label{tab:Abbr}}
	\centering
	\begin{tabu} to \textwidth [c]{X[c m] X[5 m]}
		\toprule
		\everyrow{\midrule}
		\rowfont[c]{\bfseries} Abbreviation & Explanation\\
		\taburulecolor{ltgray}
		{dB} & Decibels (a dimensionless logarithmic unit commonly used to express the intensity or power of a sound, relative to a known reference value).\\ 
		{dB \ac{spl}} & Intensity in decibels relative to 20 micropascals (μPa), an accepted standard for the threshold of human hearing \citepalias{ansi1994}.\\
		\ac{crm} & Coordinate response measure (a corpus of parallel sentences of the form “Ready ⟨\textsc{callsign}⟩ go to ⟨\textsc{color}⟩ ⟨\textsc{digit}⟩ now”).  In a typical competing speech usage, the listener is told to attend to the speech stream containing a particular callsign, and to repeat the color and digit from that stream.  See \citet{BoliaEtAl2000} for details.\\
		\fo & Fundamental frequency (of speech).  Often represented as F₀ or F0, the form \fo{} is preferred because it glyphically emphasizes the fact that the fundamental frequency is not a formant (which are typically represented as F1, F2, \etc).  The form f0 is an acceptable substitute when glyph choice is restricted to \ac{ascii} characters.\\
		F2×F1 & The acoustic space defined by the center frequencies of the first and second formants of sonorant speech sounds (usually vowels).\\
		L1, L2, \etc. & First (native) language, second language, \etc.\\
		\ac{rms} & Root mean square (a method of measuring sound power: insert formula here).\\
		\ac{snr} & Signal-to-noise ratio (insert formula here).\\
		\ac{spl} & See {dB \ac{spl}}.\\
		\everyrow{}
		\ac{srt} & Speech reception threshold (insert explanation here).\\
		\taburulecolor{black}
		\bottomrule
	\end{tabu}
\end{table}
